%%%%%%%%%%%%%%%%%%%%%%%%%%%%%%%%%%%%%%%%%%%%%%%%%%%%%%%%%%%%%%%%%%%%%%
%%  
%%  ONTARIOTECHU-THESIS.TEX
%%
%% This program can be redistributed and/or modified under the terms
%% of the LaTeX Project Public License Distributed from CTAN archives
%% in directory CTAN:/macros/latex/base/lppl.txt.
%%
%% Copyright (c) 2020 by Faisal Qureshi
%% Last Update: 2020 Feb 20
%%
%%%%%%%%%%%%%%%%%%%%%%%%%%%%%%%%%%%%%%%%%%%%%%%%%%%%%%%%%%%%%%%%%%%%%%
%%
%%  Adapted from  
%%
%%  UOIT-THESIS.TEX
%%
%% This program can be redistributed and/or modified under the terms
%% of the LaTeX Project Public License Distributed from CTAN archives
%% in directory CTAN:/macros/latex/base/lppl.txt.
%%
%% Copyright (c) 2010 by Faisal Qureshi
%% Last Update: 2010 December 20
%%
%%%%%%%%%%%%%%%%%%%%%%%%%%%%%%%%%%%%%%%%%%%%%%%%%%%%%%%%%%%%%%%%%%%%%%
%%
%% Thanks to Francois Pitt for letting me use his ut-thesis.tex and
%% ut-thesis.cls files for my purposes.
%%
%% Based upon ut-thesis.tex
%%
%% Copyright (c) 1999 by Francois Pitt
%% Last Update: 1999 May 13
%%  
%%%%%%%%%%%%%%%%%%%%%%%%%%%%%%%%%%%%%%%%%%%%%%%%%%%%%%%%%%%%%%%%%%%%%%
%%  
%%  This file is distributed in the hope that it will be useful but
%%  without any warranty (without even the implied warranty of
%%  fitness for a particular purpose).  For a description of this
%%  file's purpose, and instructions on its use, see below.
%%  
%%  Feel free to copy and redistribute this file, as long as this
%%  copyright notice remains intact and this file is distributed
%%  along with the companion file `uoit-thesis.cls'.
%%  
%%  (Thanks to Robert Bernecky for his suggestions on improving the
%%  usefulness and readability of this file.)
%%  
%%  Send all bugs, questions, comments, suggestions, etc. to the
%%  author, at <faisal.qureshi@ontariotechu.ca>.
%%  
%%%%%%%%%%%%%%%%%%%%%%%%%%%%%%%%%%%%%%%%%%%%%%%%%%%%%%%%%%%%%%%%%%%%%%
%%  
%%  To be used in conjunction with class file `ontariotechu-thesis.cls', whose
%%  features it illustrates.
%%  
%%  To comment out parts of a file, use the macro \ignore{...}
%%  around the entire block of text you want to ignore.
%%  
%%  To explicitly set the pagestyle of any inserted blank page when
%%  \cleardoublepage occurs, use one of \clearemptydoublepage or
%%  \clearplaindoublepage instead.
%%  
%%  For single-spaced quotes or quotations, use the `longquote' and
%%  `longquotation' environments.  For single-spaced, 1 1/2-spaced,
%%  or double-spaced paragraphs, use one of the environments
%%  `singlespaced', `oneandahalfspaced', or `doublespaced'.  More
%%  generally, for paragraphs with a line spacing of `n', use
%%  `\begin{newspacing}{n}...\end{newspacing}'.
%%  
%%  All other environments, commands, and options provided by the
%%  `ut-thesis' class will be described below, at the point where
%%  they should appear in the document.
%%  
%%  See the companion file `uoit-thesis.cls' for more details.
%%  
%%%%%%%%%%%%%%%%%%%%%%%%%%%%%%%%%%%%%%%%%%%%%%%%%%%%%%%%%%%%%%%%%%%%%%


%%%%%%%%%%%%         PREAMBLE         %%%%%%%%%%%%

%% Default settings format a final copy (12pt font, single-sided,
%% double-spaced, normal margins, single-spaced notes).  For a rough
%% copy (12pt font, double-sided, double-spaced, extrawide draft margins, with
%% the word "DRAFT" printed at each corner of every page), use the
%% `draft' option.  The default line spacing can be changed with one
%% of the following options: `singlespaced', `oneandahalfspaced', or
%% `doublespaced'.  The notes are always single-spaced by default, but
%% can be made to have the same spacing as the rest of the document by
%% using the option `spacednotes'.  The size of the margins can be
%% changed with one of the following options: `narrowmargins' (1 1/4"
%% left, 3/4" others), `normalmargins' (1 1/4" left, 1" others),
%% `widemargins' (1 1/4" all), `extrawidemargins' (1 1/2" all),
%% `extrawidemarginsdraft (1" left, 3" right).  Any
%% other standard option for the `report' document class can be used
%% to override the default or draft settings.

%% FINAL VERSION - UNDERGRAD THESIS
%% For "undergraduate theses", uncomment the following line and 
%% add any other options that you may require 
%\documentclass{ontariotechu-thesis}

%% DRAFT - UNDERGRAD THESIS
%% For "undergraduate draft theses", uncomment the following line and 
%% add any other options that you may require 
% \documentclass[draft]{ontariotechu-thesis}

%% FINAL VERSION - GRAD THESIS
%% For "graduate thesis" use the 'grad' option, uncomment the
%% following line and any other options that you may require
\documentclass[grad]{ontariotechu-thesis}

%% DRAFT VERSION - GRAD THESIS
%% For "graduate draft thesis" use the 'grad' option, uncomment the
%% following line and any other options that you may require
% \documentclass[draft,grad]{ontariotechu-thesis}

%% ***   Add \usepackage declarations here.   ***

\usepackage{lipsum} 
\usepackage{changepage}
\usepackage{setspace}
\usepackage{multicol}
\usepackage{graphicx}
\usepackage[export]{adjustbox}
\usepackage{subcaption}
\usepackage{rotating}
\usepackage{footnote}
\usepackage{caption}
\usepackage{amsmath,amssymb}
\usepackage{algorithm}
\usepackage{listings}
\usepackage[noend]{algpseudocode}
\usepackage[shortcuts]{extdash}
\usepackage{ctable}
\usepackage{dcolumn}
\usepackage{multirow}
\usepackage{color} 
\definecolor{codegreen}{rgb}{0,0.6,0}
\definecolor{codegray}{rgb}{0.5,0.5,0.5}
\definecolor{codepurple}{rgb}{0.58,0,0.82}
\definecolor{backcolour}{rgb}{0.95,0.95,0.92}

\lstdefinestyle{mystyle}{
    backgroundcolor=\color{backcolour},   
    commentstyle=\color{codegreen},
    keywordstyle=\color{magenta},
    numberstyle=\tiny\color{codegray},
    stringstyle=\color{codepurple},
    basicstyle=\footnotesize,
    breakatwhitespace=false,         
    breaklines=true,                 
    captionpos=b,                    
    keepspaces=false,                 
    numbers=left,                    
    numbersep=5pt,                  
    showspaces=false,                
    showstringspaces=false,
    showtabs=false,                  
    tabsize=2
}
\lstset{style=mystyle}

\DeclareMathOperator*{\argmax}{arg\,max}
\DeclareMathOperator*{\argmin}{arg\,min}
\DeclareMathOperator{\E}{\mathbb{E}}
\newcommand{\pluseq}{\mathrel{+}=}
\newcommand{\minus}{\scalebox{0.3}[0.5]{$-$}}
\newcommand\todo[1]{\textcolor{red}{#1}}

%% The line spacing of the document should be specified using one of
%% the document options given above, but if you need a line spacing
%% that is not provided by the options, you can override the default
%% line spacing for the entire document with the command
%%   `\linespacing{...}'.
%% Note that in order to get the correct appearance, the argument to
%% `\linespacing' must be equal to 1/3 + 2/3 times the desired line
%% spacing (for example, single-spaced = \linespacing{1},
%%                        1 1/2-spaced = \linespacing{1.33}, and
%%                       double-spaced = \linespacing{1.66}).

%% ***   Uncomment and fill in a value, if needed.    ***
%% ***   REMEMBER: You should NOT need to use this.  Use one of   ***
%% ***   the document class options mentionned above instead.     ***
%\linespacing{}

%%%%%%%%%%%%%%%%%%%%%%%%%%%%%%%%%%%%%%%%%%%%%%%%%%%%%%%%%%%%%%%%%%%%%%
%%                                                                  %%
%%                  ***   I M P O R T A N T   ***                   %%
%%                                                                  %%
%%  Fill in the following fields with the required information:     %%
%%   - \degree{...}       name of the degree obtained               %%
%%   - \faculty{...}      name of the faculty                       %%
%%   - \department{...}   name of the program                       %%
%%   - \gradyear{...}     year of graduation                        %%
%%   - \author{...}       name of the author                        %%
%%   - \title{...}        title of the thesis                       %%
%%%%%%%%%%%%%%%%%%%%%%%%%%%%%%%%%%%%%%%%%%%%%%%%%%%%%%%%%%%%%%%%%%%%%%

%% ***   Change this example to appropriate values.   ***
\degree{Masters of Science} % Only used when grad option is chosen
\department{Computer Science}
\faculty{Science}
\gradyear{2020}
\supervisor{Dr.~A.~Foo and Dr.~B.~Boo}
\author{Faisal Z. Qureshi}
\title{Ontario Tech U Class File Example}

%% ***   NOTE   ***
%% Put here all other formatting commands that belong in the preamble.


%% For example, to list only down to subsections in table of contents
%% (-1=part, 0=chapter, 1=section, 2=subsection, 3=subsubsection,
%%  4=paragraph, 5=subparagraph, 6=subsubparagraph).
%
\setcounter{tocdepth}{2}


%%%%%%%%%%%%      MAIN  DOCUMENT      %%%%%%%%%%%%

\begin{document}

%% ***   NOTE   ***
%% You should put all of your `\newcommand', `\newenvironment', and
%% `\newtheorem's (in other words, all the global definitions that
%% you will need throughout your thesis) in a separate file and use
%% "\input{filename}" to input it here.


%% This sets the page style and numbering for preliminary sections.
\begin{preliminary}

%% This generates the title page from the information given above.
\maketitle

%% Certificate of Approval - Start

%% The University of Ontario Institute of Technology requires the
%% Certificate of Approvale (CoA) to be included as page (ii) of
%% the PRINTED version.
%%
%% If this is required at your institution,
%% uncomment the next 3 lines.

% \cleardoublepage
% \addcontentsline{toc}{chapter}{Certificate of Approval}
% \includepdf[pages=-,pagecommand={},width=1.2\textwidth]{CoA.pdf}

% And comment out the following

\cleardoublepage
\addcontentsline{toc}{chapter}{Certificate of Approval}
\noindent \textcolor{red}{The University of Ontario Institute of Technology requires the
Certificate of Approval (CoA) to be included as page (ii) of
the PRINTED version. Check the source file on how to add
the provided CoA to your thesis.}

%% Certificate of Approval - End

%% This generates the abstract page, with the line spacing adjusted
%% according to SGS guidelines.
\addcontentsline{toc}{chapter}{Abstract}
\begin{abstract}
%% ***   Put your Abstract here.   ***
  \lipsum[1]\\
  \textcolor{red}{{\bf At most 150 words for Honors Thesis/M.Sc. or 350 words for Ph.D.}}
\vfill
\noindent {\textbf{Keywords: } \space keyword1; keyword2; keyword3; keyword4; keyword5}
\end{abstract}

%% Anything placed between the abstract and table of contents will
%% appear on a separate page since the abstract ends with \newpage
%% and the table of contents starts with \clearpage.

%% Uncomment the following to show author's declaration page.
%%
  %% \addcontentsline{toc}{chapter}{Author's Declaration}
  %% \showdeclaration

%% Uncomment the following make a statement about contributions.  Use the following
%% text as a guideline only.  You should discuss this with your supervisor.
%%
 %% \addcontentsline{toc}{chapter}{Statement of Contributions}
 %% \begin{contributions}
 %% I hereby certify that I am the sole author of this thesis and that no part of this thesis has been published. I have used standard referencing practices to acknowledge ideas, research techniques, or other materials that belong to others. Furthermore, I hereby certify that I am sole source of the creative works and/or inventive knowledge described in this thesis.
 %% \end{contributions}

%% This generates a "dedication" section, if needed.
%% (uncomment to have it appear in the document)
%%
 \addcontentsline{toc}{chapter}{Dedication}
 \begin{dedication}
 ***   Put your Dedication here.   ***

 \lipsum[2]
 \end{dedication}

%% This generates an "acknowledgements" section, if needed.
%% (uncomment to have it appear in the document)
%%
\addcontentsline{toc}{chapter}{Acknowledgment}
\begin{acknowledgements}
***   Put your Acknowledgements here.   ***

\lipsum[2]
\end{acknowledgements}

%% This generates the Table of Contents (on a separate page).
\tableofcontents

%% This generates the List of Tables (on a separate page), if needed.
%% (uncomment to have it appear in the document)
\listoftables

%% This generates the List of Figures (on a separate page), if needed.
%% (uncomment to have it appear in the document)
\listoffigures

%% End of the preliminary sections: reset page style and numbering.
\end{preliminary}

%%%%%%%%%%%%%%%%%%%%%%%%%%%%%%%%%%%%%%%%%%%%%%%%%%%%%%%%%%%%%%%%%%%%%%
%%  Put your Chapters here; the easiest way to do this is to keep   %%
%%  each chapter in a separate file and `\include' all the files    %%
%%  right here.  Note that each chapter file should start with the  %%
%%  line "\chapter{ChapterName}".  Note that using `\include'       %%
%%  instead of `\input' makes each chapter start on a new page.     %%
%%%%%%%%%%%%%%%%%%%%%%%%%%%%%%%%%%%%%%%%%%%%%%%%%%%%%%%%%%%%%%%%%%%%%%

%% ***   Include chapter files here.   ***

\chapter{The First Chapter}
\section{The First Section}

And you must read~\cite{Str00} at least twice before you start writing your thesis.

\lipsum[2-4]

\chapter{The Second Chapter}
\section{A Section in the Second Chapter}
\lipsum[5-6]


%% This adds a line for the Bibliography in the Table of Contents.
\addcontentsline{toc}{chapter}{Bibliography}
%% ***   Set the bibliography style.   ***
%% (change according to your preference)
\bibliographystyle{plain}
%% ***   Set the bibliography file.   ***
%% ("thesis.bib" by default; change if needed)
\bibliography{thesis}

%% ***   NOTE   ***
%% If you don't use bibliography files, comment out the previous line
%% and use \begin{thebibliography}...\end{thebibliography}.  (In that
%% case, you should probably put the bibliography in a separate file
%% and `\include' or `\input' it here).

\end{document}

%%%%%%%%%%%%%%%%%%%%%%%%%%%%%%%%%%%%%%%%%%%%%%%%%%%%%%%%%%%%%%%%%%%%%%
%%  End of ONTARIOTECHU-THESIS.TEX
%%%%%%%%%%%%%%%%%%%%%%%%%%%%%%%%%%%%%%%%%%%%%%%%%%%%%%%%%%%%%%%%%%%%%%
